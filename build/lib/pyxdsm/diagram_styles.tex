\usetikzlibrary{backgrounds}


% Define deep colors from Seaborn in LaTeX
% Seaborn Deep Colors
\definecolor{deepBlue}{HTML}{4C72B0}   % Replacing blue
\definecolor{deepGreen}{HTML}{55A868}  % Replacing green
\definecolor{deepRed}{HTML}{C44E52}    % Replacing red
\definecolor{deepPurple}{HTML}{8172B2} % Replacing cyan
\definecolor{deepYellow}{HTML}{CCB974} % Replacing yellow
\definecolor{deepCyan}{HTML}{64B5CD}   % Replacing salmon
\definecolor{deepGray}{HTML}{4C4C4C}       % Additional gray
\definecolor{deepPink}{HTML}{D06666}       % Additional pink
\definecolor{deepBrown}{HTML}{8D7B7B}      % Additional brown
\definecolor{deepLavender}{HTML}{7A6F89}   % Additional lavender
\definecolor{lightGray}{rgb}{0.2, 0.2, 0.2} % Lighter gray (RGB: 85% brightness)


\tikzset{
  RoundedSquare/.style={
    rectangle, % Base shape
    rounded corners=5pt, % Adjust corner radius
    draw, % Outline
    fill=deepGray!\fillOpacity, % Fill color with opacity
    inner sep=6pt, % Padding inside the shape
    minimum size=1cm, % Ensures it's a square
    text badly centered
  }
}

\tikzstyle{every node}=[font=\sffamily,align=center]

\newcommand{\fillOpacity}{90}
\newcommand{\fillOpacityTwo}{30}
\newcommand{\fillOpacityThree}{10}

% Component shapes
\newcommand{\compShape}{rectangle}
\newcommand{\groupShape}{RoundedSquare}
\newcommand{\funcShape}{chamfered rectangle}
\newcommand{\procShape}{rounded rectangle}

% Colors - updated with Seaborn deep colors
\newcommand{\explicitColor}{deepGreen}
\newcommand{\customColor}{deepYellow}
\newcommand{\implicitColor}{deepCyan}
\newcommand{\optimizationColor}{lightGray}
\newcommand{\fluidColor}{deepBlue}
\newcommand{\solidColor}{deepRed}
\newcommand{\propulsionColor}{deepYellow}
\newcommand{\FuncFluidColor}{deepBlue}
\newcommand{\FuncSolidColor}{deepRed}

% Component types - updated styles with Seaborn deep colors
\tikzstyle{Optimization} = [\procShape,draw,fill=\optimizationColor!\fillOpacity,text=white,inner sep=6pt,minimum height=1cm,text badly centered]

\tikzstyle{MDA} = [\procShape,draw,fill=deepYellow!\fillOpacityTwo,inner sep=6pt,minimum height=1cm,text badly centered]

\tikzstyle{FLUID} = [\groupShape,draw,fill=deepBlue!\fillOpacity,inner sep=6pt,minimum height=1cm,text badly centered]
\tikzstyle{SOLID} = [\procShape,draw,fill=deepRed!\fillOpacity,inner sep=6pt,minimum height=1cm,text badly centered]
\tikzstyle{PROPULSION} = [\procShape,draw,fill=deepGreen!\fillOpacity,inner sep=6pt,minimum height=1cm,text badly centered]

\tikzstyle{DOE} = [\procShape,draw,fill=\optimizationColor!\fillOpacity,inner sep=6pt,minimum height=1cm,text badly centered]

\tikzstyle{SubOptimization} = [\groupShape,draw,fill=\optimizationColor!\fillOpacity,inner sep=6pt,minimum height=1cm,text badly centered]

\tikzstyle{Group} = [\groupShape,draw,fill=\customColor!\fillOpacity,inner sep=6pt,minimum height=1cm,text badly centered]
\tikzstyle{ImplicitGroup} = [\groupShape,draw,fill=\implicitColor!\fillOpacity,inner sep=6pt,minimum height=1cm,text badly centered]

%% FLuid domain function
\tikzstyle{FluidFunction} = [\funcShape,draw,fill=\FuncFluidColor!\fillOpacityTwo,inner sep=6pt,minimum height=1cm,text badly centered]

%% Solid domain function
\tikzstyle{SolidFunction} = [\funcShape,draw,fill=\FuncSolidColor!\fillOpacityTwo,inner sep=6pt,minimum height=1cm,text badly centered]


\tikzstyle{Metamodel} = [\compShape,draw,fill=deepYellow!\fillOpacity,inner sep=6pt,minimum height=1cm,text badly centered]

%% A simple command to give the repeated structure look for components and data
\tikzstyle{stack} = [double copy shadow={shadow xshift=.75ex, shadow yshift=-.75ex}]
%% A simple command to fade components and data, e.g., demonstrating a sequence of steps in an animation
\tikzstyle{faded} = [draw=black!10,fill=white,text opacity=0.2]

%% Simple fading commands for the lines
\tikzstyle{fadeddata} = [color=black!20]
\tikzstyle{fadedprocess} = [color=black!50]

% Data types
\newcommand{\dataRightAngle}{105}
\newcommand{\dataLeftAngle}{75}

\tikzstyle{DataInter} = [trapezium,trapezium left angle=\dataLeftAngle,trapezium right angle=\dataRightAngle,draw,fill=black!10]
\tikzstyle{DataIO} = [trapezium,trapezium left angle=\dataLeftAngle,trapezium right angle=\dataRightAngle,draw,fill=white]

% Edges
\tikzstyle{DataLine} = [color=black!40,line width=5pt,line cap=rect]
\tikzstyle{ProcessHV} = [-,line width=1pt,to path={-| (\tikztotarget)}]
\tikzstyle{ProcessHVA} = [->,line width=1pt,to path={-| (\tikztotarget)}]
\tikzstyle{ProcessTip} = [-,line width=1pt]
\tikzstyle{ProcessTipA} = [->, line width=1pt]
\tikzstyle{FadedProcessHV} = [-,line width=1pt,to path={-| (\tikztotarget)},color=black!30]
\tikzstyle{FadedProcessHVA} = [->,line width=1pt,to path={-| (\tikztotarget)},color=black!30]
\tikzstyle{FadedProcessTip} = [-,line width=1pt,color=black!30]
\tikzstyle{FadedProcessTipA} = [->, line width=1pt,color=black!30]

% Matrix options
\tikzstyle{MatrixSetup} = [row sep=3mm, column sep=2mm]

% Declare a background layer for showing node connections
\pgfdeclarelayer{data}
\pgfdeclarelayer{process}
\pgfsetlayers{data,process,main}

% Commands to split component text over multiple lines or columns
\newcommand{\MultilineComponent}[2]
{
	\begin{minipage}{#1}
	\begin{center}
		#2
	\end{center}
	\end{minipage}
}

\newcommand{\TwolineComponent}[3]
{
	\begin{minipage}{#1}
	\begin{center}
		#2 \linebreak #3
	\end{center}
	\end{minipage}
}

\newcommand{\ThreelineComponent}[4]
{
	\begin{minipage}{#1}
	\begin{center}
		#2 \linebreak #3 \linebreak #4
	\end{center}
	\end{minipage}
}

\newcommand{\MultiColumnComponent}[5]
{
	\begin{minipage}{#1}
	\begin{center}
	#2 \linebreak #3
	\end{center}
	\begin{minipage}{0.49\textwidth}
	\begin{center}
	#4
	\end{center}
	\end{minipage}
	\begin{minipage}{0.49\textwidth}
	\begin{center}
	#5
	\end{center}
	\end{minipage}
	\end{minipage}
}

\def\arraystretch{1.3}
